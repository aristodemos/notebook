% TEMPLATE for Usenix papers, specifically to meet requirements of
%  USENIX '05
% originally a template for producing IEEE-format articles using LaTeX.
%   written by Matthew Ward, CS Department, Worcester Polytechnic Institute.
% adapted by David Beazley for his excellent SWIG paper in Proceedings,
%   Tcl 96
% turned into a smartass generic template by De Clarke, with thanks to
%   both the above pioneers
% use at your own risk.  Complaints to /dev/null.
% make it two column with no page numbering, default is 10 point

% Munged by Fred Douglis <douglis@research.att.com> 10/97 to separate
% the .sty file from the LaTeX source template, so that people can
% more easily include the .sty file into an existing document.  Also
% changed to more closely follow the style guidelines as represented
% by the Word sample file.

% Note that since 2010, USENIX does not require endnotes. If you want
% foot of page notes, don't include the endnotes package in the
% usepackage command, below.

% This version uses the latex2e styles, not the very ancient 2.09 stuff.
\documentclass[letterpaper,twocolumn,10pt]{article}
\usepackage{usenix,epsfig,endnotes}
\begin{document}

%don't want date printed
\date{}

%make title bold and 14 pt font (Latex default is non-bold, 16 pt)
\title{\Large \bf Distributed File Systems}

%for single author (just remove % characters)
\author{
{\rm Aris\ Paphitis}\\
CUT
% copy the following lines to add more authors
% \and
% {\rm Name}\\
%Name Institution
} % end author

\maketitle

% Use the following at camera-ready time to suppress page numbers.
% Comment it out when you first submit the paper for review.
\thispagestyle{empty}


\subsection*{Key Point}
Distributed File Systems - NFS, GFS, HDFS

\section{Network File System -NFS}
Developed by Sun Microsystems, used in all modern Linux systems to join the file systems on separate computers into a logical whole.
The basic idea behinf NFS is to allow a group of clients and servers to share a common file system.
Each NFS server exports one or more of its directories for access by remote clients.Clients acess exported directories by mounting them. When a client mounts a remote directory, it becomes part of its directory hierarchy. The mount point is entirely local to the clients; the server does not know where it is mounted on any of its clients.
To support heterogeneity, NFS defines an interface between clients and servers with two protocols. The first handles mounting. The second is for directory and file access. Clients send messages to servers to manipulate directories and read/write files; but they cannot \textit{open} or \textit{close} files.



{\footnotesize \bibliographystyle{acm}
\bibliography{usenix}}


\theendnotes

\end{document}
